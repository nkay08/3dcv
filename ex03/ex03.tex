\documentclass[a4paper, twoside, english]{article}

\usepackage{amsmath}
\usepackage{amsfonts}
%\usepackage{ihci}
\usepackage{graphicx}
\usepackage{subfig}


\title{Exercise 3 \\ 3D Computer Vision}  % Replace "Template Report" with Exercise 1, Exercise 2, etc
\author{Niklas Klipphahn \\ Matrikel-Nr : 384796
\and
Udit Dokania \\ Matrikel-Nr : 407123
\and
Khurram Azeem Hashmi \\ Matrikel-Nr : 406982
}                       % Replace with your names
   
   
   
\begin{document}
	
\maketitle

\section{Homography Definition}
$
\begin{bmatrix}
	x_1'\\
	x_2'\\
	\vdots\\
	v_{n+1}
\end{bmatrix}
= 
\begin{bmatrix}
	R_{11} & R_{12} & \dots & R_{1,n+1}\\
	R_{21} & \ddots \\
	\vdots \\
	R_{n+1,1} & \dots & \dots & R_{n+1,n+1}
\end{bmatrix}
$
or $x' = Hx$ with $(n+1)^2 -1$ DOF

\section{Line preservation of homographies}

Let $x_1,x_2,x_3 \in \mathbb{P}$ be three points on a line.
The definition of aline should hold for every point with the same $l=(a,b,c)$\\
$l x_1 = 0, l x_2 = 0, l x_3 = 0$\\

\begin{tabular}{lll}
$H x_1 = h(x_1)$ & &$ l h(x_1)=0$\\
$H x_2 = h(x_2)$ & $\rightarrow$ & $l h(x2) = 0$\\
$H x_3 = h(x_3)$ & & $l h(x_3) = 0$	
\end{tabular}


\end{document}